\documentclass{article}
\usepackage[T1]{fontenc}
\usepackage[cm-default]{fontspec}
\usepackage{lmodern}

\usepackage{amssymb,amsmath}
\usepackage{ifxetex,ifluatex}
\usepackage{sectsty}
\usepackage{ctable}
\usepackage{hyperref}
\usepackage[numbers]{natbib}

\begin{document}

% TODO:
%   * Explain that every SEQC sample was put in the same experiment for
%     isolator.


\section{Bias Correction}\label{bias-correction}

Isolator attempts to model and correct for multiple factors that can
conflate the transcript expression estimates.

\subsection{3' bias}\label{bias}

Selection of polyadenylated transcripts is a common step mRNA-Seq used
to avoid sequencing introns, partially degraded transcripts, and
ribosomal RNA.

To model the effect we fit a one parameter model in which a transcript
is truncated at any position with probability $p$. The probability of
observing a fragment at a position $k$ of length $\ell$, counting from
the 3' end, is then proportional to the probability that the transcript
was \emph{not} truncated before $k + \ell$, which is $(1-p)^{k + \ell}$.
When fit, $p$ is typically quite small, on the order of $10^{-5}$, in
which case this correction has little effect on transcripts shorter that
several kilobases.

\subsection{Fragmentation effects}\label{fragmentation-effects}

Popular RNA-Seq protocols such as Illumina's TruSeq typically involve a
random fragmentation step. Subtle implications arise from this process.

Existing statisticaly models (see {[}@Pachter:2011wm{]} for a review),
assume fragments are sampled uniformly at random from a transcript. This
does not exactly match the implications of random fragmentation in
which, under an ideal model, \emph{breakpoints} rather than fragments
are introduced uniformly at random. For a fragment to be observed it
must pass size selection and have fallen between two breakpoints, or one
breakpoint and the end of the transcript. Since the ends of a transcript
act as fixed breakpoints, the result isAsome enrichment of fragments at
either end of the transcript.

\subsection{Fragment GC content}\label{fragment-gc-content}

\section{Analysis}

All reads were aligned using the STAR aligner \cite{Dobin:2013fg}. RNA-Seq
methods vary in their expected input: some work on raw reads, others require
either alignments relative to genome or relative to transcript sequences. STAR
allows the same alignments to be output in transcript and genome coordinates, so
we were able to use the same alignments for each method, excluding the alignment
free methods. Quantification was performed against version 76 of the Ensembl
annotations and revision 38 of the human genome sequence.

\begin{table}
\begin{center}
\begin{tabular}[c]{ll}
\toprule\addlinespace
Method & Version \\
\midrule
STAR & 2.4.0k \\\addlinespace
Cufflinks & 2.2.1 \\\addlinespace
BitSeq & 0.7.0 \\\addlinespace
RSEM & 1.2.19 \\\addlinespace
Sailfish & 0.6.3 \\\addlinespace
eXpress & 1.5.1 \\\addlinespace
Kallisto & 0.42 \\\addlinespace
Salmon & 0.4.1 \\\addlinespace
\bottomrule
\addlinespace
\end{tabular}
\caption{Versions of software used in the analysis.}
\end{center}
\end{table}

\begin{table}
\begin{center}
\begin{tabular}[c]{@{}lrr@{}}
\toprule\addlinespace
Method & $c$ vs $0.75a + 0.25b$ & $d$ vs $0.25a + 0.75b$
\\\addlinespace
\midrule
Isolator & \textbf{0.995} & \textbf{0.995}
\\\addlinespace
BitSeq & 0.994 & 0.994
\\\addlinespace
RSEM/PM & 0.994 & 0.993
\\\addlinespace
RSEM/ML & 0.987 & 0.987
\\\addlinespace
Salmon & 0.986 & 0.986
\\\addlinespace
eXpress & 0.983 & 0.983
\\\addlinespace
Sailfish & 0.978 & 0.976
\\\addlinespace
Kallisto & 0.972 & 0.970
\\\addlinespace
Cufflinks & 0.941 & 0.982
\\\addlinespace
\bottomrule
\addlinespace
\end{tabular}
\caption{Proportionality correlation between gene-level estimates for
the mixed samples C and D and weighted averages of estimates for A and
B, corresponding to the mixture proportions for C and D. TODO: move this
table to the supplement.}
\end{center}
\label{table:seqcgenes}
\end{table}

\begin{figure}
\begin{center}
\includegraphics[width=0.7\textwidth]{analysis/seqc/runtime.pdf}
\caption{Run time needed to process the SEQC data presented in the Results
sections. All methods were run a Google Compute Engine instance backed by four
Intel Xeon cores and 52GB of a memory.}
\end{center}
\label{fig:runtime}
\end{figure}

\section{Analysis of cardiomyocyte data}

Different methods often produce very different estimates of isoform abundance.
To demonstrate the effect this has on downstream analysis, we analyzed the
cardiomyocyte data from \cite{Kuppusamy:2015ey} using each method compared in
the main paper.

\subsection{qPCR Validation}\label{cardioqpcr}

Total RNAs for 2 independent HFA and HFV samples were extracted using RNAeasy
mini kit (Qiagen) (6). Two independent HAH total RNA samples were purchased
commercially from Clontech and Agilent. Total RNAs for 3 independent replicates
of Day 20 cardiomyocytes (derived from human embryonic stem cells) were
extracted with miRNeasy mini kit (Qiagen). All RNA samples were further DNAse
treated with DNA-free kit (Ambion) and reverse transcription reaction was
carried out using Omniscript RT kit (Qiagen). Gene specific primers were
designed targeting the exon-exon junction for specific exons of interest using
the freely available software APE. The primer sequences were as follows:

\begin{center}
\begin{tabular}{rl}
TNNT2   & F-CAGAGACCATGTCTGACATAGA \\
        & R-CTGCCTCCTCCTGCTCGTx \\
OBSL1   & F-CCTGTACGCTGGTACAAGGA \\
        & R-CTCTTCCACGCTGACAAGGA \\
LTBP4   & F-CAGCCCAAAAAGTGTGCAGG \\
        & R-CTGTCTCTTCCTCACGCTGG \\
OSBPL1A & F-TCACAGCTCTGCTCAACAGG \\
        & R-ACATTGTTTATGGGCCCGGT \\
ACOT9   & F-CCACATTCATGAAGCATGTTCA \\
        & R-CCTACTATCTCCCGCAACT \\
IMMT    & F-TGCACTTTCAGAAGAAGCATCC \\
        & R-TTTTCCTGTTGTGCAAGGCG \\
PALM    & F-TGAAGGCAGCCATGTACTCG \\
        & R-GGACCACTTTGGTCTCGTCC
\end{tabular}
\end{center}

\begin{figure}
\includegraphics[width=0.5\textwidth]{analysis/cardio/ACOT9-comparison.pdf}
\includegraphics[width=0.5\textwidth]{analysis/cardio/IMMT-comparison.pdf}
\includegraphics[width=0.5\textwidth]{analysis/cardio/LTBP4-comparison.pdf}
\includegraphics[width=0.5\textwidth]{analysis/cardio/OBSL1-comparison.pdf}
\includegraphics[width=0.5\textwidth]{analysis/cardio/OSBPL1A-comparison.pdf}
\includegraphics[width=0.5\textwidth]{analysis/cardio/PALM-comparison.pdf}
\includegraphics[width=0.5\textwidth]{analysis/cardio/TNNT2-comparison.pdf}
\caption{Point estimates of fold change of exon expression from qPCR and and
RNA-Seq using various methods. Exon expression for RNA-Seq is computed by
estimating transcript expression, then summing the expression of all transcripts
that share a parcitular exon.}
\label{fig:cardiofoldchange}
\end{figure}

\subsection{Principal Component Analysis}\label{cardiopca}

A simple analysis often peformed on gene expression data is principal component
analysis (PCA). Due to large disagreements in isoform abundance estimates, each
method produces qualitatively different results. There is no ground truth for
this data, but the results from Isolator in some ways fit our understanding of
the data more closely than the other methods. Uniquely in the Isolator data,
fetal atrium and fetal ventricle are grouped very closely and clearly separated
from the H7 day20 data.

In most methods the two adult samples are quite heterogenous. The exceptions are
Kallisto and eXpress. While those both show closer agreement between the adult
samples, the fetal samples are far more spread out.

Though Isolator uses a model with informative priors, it is important to note
that the fact that fetal atrium and ventricle tissue in particular are similar
is not something that is encoded is this model.

\begin{figure}
\includegraphics[width=0.5\textwidth]{analysis/cardio/isolator-pca.pdf}
\includegraphics[width=0.5\textwidth]{analysis/cardio/salmon-pca.pdf}
\includegraphics[width=0.5\textwidth]{analysis/cardio/sailfish-pca.pdf}
\includegraphics[width=0.5\textwidth]{analysis/cardio/kallisto-pca.pdf}
\includegraphics[width=0.5\textwidth]{analysis/cardio/express-pca.pdf}
\includegraphics[width=0.5\textwidth]{analysis/cardio/cufflinks-pca.pdf}
\includegraphics[width=0.5\textwidth]{analysis/cardio/pm-rsem-pca.pdf}
\includegraphics[width=0.5\textwidth]{analysis/cardio/ml-rsem-pca.pdf}
\caption{The first two principal components from principal component analysis
applied to transcript expression estimates using a variety of methods.}
\label{fig:cardiopca}
\end{figure}

\end{document}
